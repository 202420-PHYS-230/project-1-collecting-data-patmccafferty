% Options for packages loaded elsewhere
\PassOptionsToPackage{unicode}{hyperref}
\PassOptionsToPackage{hyphens}{url}
\PassOptionsToPackage{dvipsnames,svgnames,x11names}{xcolor}
%
\documentclass[
  letterpaper,
  DIV=11,
  numbers=noendperiod]{scrartcl}

\usepackage{amsmath,amssymb}
\usepackage{iftex}
\ifPDFTeX
  \usepackage[T1]{fontenc}
  \usepackage[utf8]{inputenc}
  \usepackage{textcomp} % provide euro and other symbols
\else % if luatex or xetex
  \usepackage{unicode-math}
  \defaultfontfeatures{Scale=MatchLowercase}
  \defaultfontfeatures[\rmfamily]{Ligatures=TeX,Scale=1}
\fi
\usepackage{lmodern}
\ifPDFTeX\else  
    % xetex/luatex font selection
\fi
% Use upquote if available, for straight quotes in verbatim environments
\IfFileExists{upquote.sty}{\usepackage{upquote}}{}
\IfFileExists{microtype.sty}{% use microtype if available
  \usepackage[]{microtype}
  \UseMicrotypeSet[protrusion]{basicmath} % disable protrusion for tt fonts
}{}
\makeatletter
\@ifundefined{KOMAClassName}{% if non-KOMA class
  \IfFileExists{parskip.sty}{%
    \usepackage{parskip}
  }{% else
    \setlength{\parindent}{0pt}
    \setlength{\parskip}{6pt plus 2pt minus 1pt}}
}{% if KOMA class
  \KOMAoptions{parskip=half}}
\makeatother
\usepackage{xcolor}
\setlength{\emergencystretch}{3em} % prevent overfull lines
\setcounter{secnumdepth}{-\maxdimen} % remove section numbering
% Make \paragraph and \subparagraph free-standing
\makeatletter
\ifx\paragraph\undefined\else
  \let\oldparagraph\paragraph
  \renewcommand{\paragraph}{
    \@ifstar
      \xxxParagraphStar
      \xxxParagraphNoStar
  }
  \newcommand{\xxxParagraphStar}[1]{\oldparagraph*{#1}\mbox{}}
  \newcommand{\xxxParagraphNoStar}[1]{\oldparagraph{#1}\mbox{}}
\fi
\ifx\subparagraph\undefined\else
  \let\oldsubparagraph\subparagraph
  \renewcommand{\subparagraph}{
    \@ifstar
      \xxxSubParagraphStar
      \xxxSubParagraphNoStar
  }
  \newcommand{\xxxSubParagraphStar}[1]{\oldsubparagraph*{#1}\mbox{}}
  \newcommand{\xxxSubParagraphNoStar}[1]{\oldsubparagraph{#1}\mbox{}}
\fi
\makeatother

\usepackage{color}
\usepackage{fancyvrb}
\newcommand{\VerbBar}{|}
\newcommand{\VERB}{\Verb[commandchars=\\\{\}]}
\DefineVerbatimEnvironment{Highlighting}{Verbatim}{commandchars=\\\{\}}
% Add ',fontsize=\small' for more characters per line
\usepackage{framed}
\definecolor{shadecolor}{RGB}{241,243,245}
\newenvironment{Shaded}{\begin{snugshade}}{\end{snugshade}}
\newcommand{\AlertTok}[1]{\textcolor[rgb]{0.68,0.00,0.00}{#1}}
\newcommand{\AnnotationTok}[1]{\textcolor[rgb]{0.37,0.37,0.37}{#1}}
\newcommand{\AttributeTok}[1]{\textcolor[rgb]{0.40,0.45,0.13}{#1}}
\newcommand{\BaseNTok}[1]{\textcolor[rgb]{0.68,0.00,0.00}{#1}}
\newcommand{\BuiltInTok}[1]{\textcolor[rgb]{0.00,0.23,0.31}{#1}}
\newcommand{\CharTok}[1]{\textcolor[rgb]{0.13,0.47,0.30}{#1}}
\newcommand{\CommentTok}[1]{\textcolor[rgb]{0.37,0.37,0.37}{#1}}
\newcommand{\CommentVarTok}[1]{\textcolor[rgb]{0.37,0.37,0.37}{\textit{#1}}}
\newcommand{\ConstantTok}[1]{\textcolor[rgb]{0.56,0.35,0.01}{#1}}
\newcommand{\ControlFlowTok}[1]{\textcolor[rgb]{0.00,0.23,0.31}{\textbf{#1}}}
\newcommand{\DataTypeTok}[1]{\textcolor[rgb]{0.68,0.00,0.00}{#1}}
\newcommand{\DecValTok}[1]{\textcolor[rgb]{0.68,0.00,0.00}{#1}}
\newcommand{\DocumentationTok}[1]{\textcolor[rgb]{0.37,0.37,0.37}{\textit{#1}}}
\newcommand{\ErrorTok}[1]{\textcolor[rgb]{0.68,0.00,0.00}{#1}}
\newcommand{\ExtensionTok}[1]{\textcolor[rgb]{0.00,0.23,0.31}{#1}}
\newcommand{\FloatTok}[1]{\textcolor[rgb]{0.68,0.00,0.00}{#1}}
\newcommand{\FunctionTok}[1]{\textcolor[rgb]{0.28,0.35,0.67}{#1}}
\newcommand{\ImportTok}[1]{\textcolor[rgb]{0.00,0.46,0.62}{#1}}
\newcommand{\InformationTok}[1]{\textcolor[rgb]{0.37,0.37,0.37}{#1}}
\newcommand{\KeywordTok}[1]{\textcolor[rgb]{0.00,0.23,0.31}{\textbf{#1}}}
\newcommand{\NormalTok}[1]{\textcolor[rgb]{0.00,0.23,0.31}{#1}}
\newcommand{\OperatorTok}[1]{\textcolor[rgb]{0.37,0.37,0.37}{#1}}
\newcommand{\OtherTok}[1]{\textcolor[rgb]{0.00,0.23,0.31}{#1}}
\newcommand{\PreprocessorTok}[1]{\textcolor[rgb]{0.68,0.00,0.00}{#1}}
\newcommand{\RegionMarkerTok}[1]{\textcolor[rgb]{0.00,0.23,0.31}{#1}}
\newcommand{\SpecialCharTok}[1]{\textcolor[rgb]{0.37,0.37,0.37}{#1}}
\newcommand{\SpecialStringTok}[1]{\textcolor[rgb]{0.13,0.47,0.30}{#1}}
\newcommand{\StringTok}[1]{\textcolor[rgb]{0.13,0.47,0.30}{#1}}
\newcommand{\VariableTok}[1]{\textcolor[rgb]{0.07,0.07,0.07}{#1}}
\newcommand{\VerbatimStringTok}[1]{\textcolor[rgb]{0.13,0.47,0.30}{#1}}
\newcommand{\WarningTok}[1]{\textcolor[rgb]{0.37,0.37,0.37}{\textit{#1}}}

\providecommand{\tightlist}{%
  \setlength{\itemsep}{0pt}\setlength{\parskip}{0pt}}\usepackage{longtable,booktabs,array}
\usepackage{calc} % for calculating minipage widths
% Correct order of tables after \paragraph or \subparagraph
\usepackage{etoolbox}
\makeatletter
\patchcmd\longtable{\par}{\if@noskipsec\mbox{}\fi\par}{}{}
\makeatother
% Allow footnotes in longtable head/foot
\IfFileExists{footnotehyper.sty}{\usepackage{footnotehyper}}{\usepackage{footnote}}
\makesavenoteenv{longtable}
\usepackage{graphicx}
\makeatletter
\newsavebox\pandoc@box
\newcommand*\pandocbounded[1]{% scales image to fit in text height/width
  \sbox\pandoc@box{#1}%
  \Gscale@div\@tempa{\textheight}{\dimexpr\ht\pandoc@box+\dp\pandoc@box\relax}%
  \Gscale@div\@tempb{\linewidth}{\wd\pandoc@box}%
  \ifdim\@tempb\p@<\@tempa\p@\let\@tempa\@tempb\fi% select the smaller of both
  \ifdim\@tempa\p@<\p@\scalebox{\@tempa}{\usebox\pandoc@box}%
  \else\usebox{\pandoc@box}%
  \fi%
}
% Set default figure placement to htbp
\def\fps@figure{htbp}
\makeatother

\KOMAoption{captions}{tableheading}
\makeatletter
\@ifpackageloaded{caption}{}{\usepackage{caption}}
\AtBeginDocument{%
\ifdefined\contentsname
  \renewcommand*\contentsname{Table of contents}
\else
  \newcommand\contentsname{Table of contents}
\fi
\ifdefined\listfigurename
  \renewcommand*\listfigurename{List of Figures}
\else
  \newcommand\listfigurename{List of Figures}
\fi
\ifdefined\listtablename
  \renewcommand*\listtablename{List of Tables}
\else
  \newcommand\listtablename{List of Tables}
\fi
\ifdefined\figurename
  \renewcommand*\figurename{Figure}
\else
  \newcommand\figurename{Figure}
\fi
\ifdefined\tablename
  \renewcommand*\tablename{Table}
\else
  \newcommand\tablename{Table}
\fi
}
\@ifpackageloaded{float}{}{\usepackage{float}}
\floatstyle{ruled}
\@ifundefined{c@chapter}{\newfloat{codelisting}{h}{lop}}{\newfloat{codelisting}{h}{lop}[chapter]}
\floatname{codelisting}{Listing}
\newcommand*\listoflistings{\listof{codelisting}{List of Listings}}
\makeatother
\makeatletter
\makeatother
\makeatletter
\@ifpackageloaded{caption}{}{\usepackage{caption}}
\@ifpackageloaded{subcaption}{}{\usepackage{subcaption}}
\makeatother

\usepackage{bookmark}

\IfFileExists{xurl.sty}{\usepackage{xurl}}{} % add URL line breaks if available
\urlstyle{same} % disable monospaced font for URLs
\hypersetup{
  pdftitle={Collecting Data: Project 1},
  colorlinks=true,
  linkcolor={blue},
  filecolor={Maroon},
  citecolor={Blue},
  urlcolor={Blue},
  pdfcreator={LaTeX via pandoc}}


\title{Collecting Data: Project 1}
\author{}
\date{}

\begin{document}
\maketitle


\section{Template}\label{template}

\subsection{(Delete this section when you commit your
report)}\label{delete-this-section-when-you-commit-your-report}

For each project there'll be a template like this. I'll try to make the
early templates more of a ``fill in the blank'' experience, but as the
semester progresses you'll be seeing less and less scaffolding and more
of bare bones layout with a lot of room for creativity and flexibility.
The rubric will remain consistent for all the projects. When you are
ready to commit your final report, you can delete this cell, and any
other cells that are purely made up of instructions or guidance, and
leave only the cells that belong in the final report.

This doument is a \href{https://quarto.org}{Quarto} markdown document.
We're using Quarto in this class to prepare project reports. It has some
nice features that make it well suited to this sort of task. For example
it's easy to pull a graph from a notebook like this:

\begin{figure}

{\centering \pandocbounded{\includegraphics[keepaspectratio]{index_files/figure-latex/ExampleLinearFit-curve_fit_example-output-1.png}}

}

\caption{Example of curve fitting with
\texttt{scipy.optimize.curve\_fit}.}

\end{figure}%

It's easy to embed math, like this: \(x=\sin(3t + \phi)\), right in your
prose. You can also embed images that you've created and saved:

\begin{figure}[H]

{\centering \includegraphics[width=5.80208in,height=\textheight,keepaspectratio]{images/divider.png}

}

\caption{Fig. 1: The external circuit we used to implement a voltage
divider and an RC circuit.}

\end{figure}%

Further, if you have a table of data that makes sense to include, that's
also easy to do:

\phantomsection\label{read_data}
\begin{longtable}[]{@{}lll@{}}
\toprule\noalign{}
& t & y \\
\midrule\noalign{}
\endhead
\bottomrule\noalign{}
\endlastfoot
0 & 0.000000 & 0.970921 \\
1 & 0.263158 & 1.686714 \\
2 & 0.526316 & 2.199626 \\
3 & 0.789474 & 2.516062 \\
4 & 1.052632 & 3.067824 \\
\end{longtable}

Loading data from a csv file

The report should have 5 sections:

\begin{enumerate}
\def\labelenumi{\arabic{enumi}.}
\tightlist
\item
  Abstract and Description

  \begin{itemize}
  \tightlist
  \item
    An abstract is a very brief statment of what was measured and the
    results produced. In the best case, there should be an estimate of
    uncertainty as well.
  \item
    The description is a more in-depth explanation of the experiments
    done, but without the result. This is not a ``procedure,'' but
    simply an explanation of what was accomplished, and roughly how it
    was done.
  \end{itemize}
\item
  Schematic and Background

  \begin{itemize}
  \tightlist
  \item
    A circuit schematic of each experiment.
  \item
    Any background information needed to understand the circuits.
  \item
    Any simulations you performed to prepare for measurements
  \end{itemize}
\item
  Function/Data Collection

  \begin{itemize}
  \tightlist
  \item
    This is where you explain how you know the circuit functioned
    correctly.
  \item
    What data did you collect?
  \item
    How did you validate it?
  \end{itemize}
\item
  Results

  \begin{itemize}
  \tightlist
  \item
    Analysis of data collected in part 3 to infer the values from the
    abstract.
  \item
    If there are statistical exercises in this project, you'll post the
    answers as a part of this section. Please state the question, and
    provide the solution.
  \end{itemize}
\item
  Conclusion

  \begin{itemize}
  \tightlist
  \item
    A summary of your inferences.
  \end{itemize}
\end{enumerate}

Now it's time for the actual report. Delete everything above the line
below, including the line! Within the report itself, all you need to do
is to perform the actual experiments and replace both the data, and the
results, with the actual data you collect, and the actual results you
produce using that data. This week most of the work of writing, coding,
and analyzing has already been done. All you need to do is to replace
the sample data with real data. This is deliberately done so you'll have
a concrete example of how to proceed in future weeks.

\begin{center}\rule{0.5\linewidth}{0.5pt}\end{center}

\section{1. Abstract and Description}\label{abstract-and-description}

\subsection{Abstract}\label{abstract}

We measured the voltage at two points in a resistor network to validate
the behavior of the circuit. We found the ratio of voltages was:
\(0.665\pm 0.001\) close to the expected value of \(0.667\). We also
measured the decay over time of the voltage on a capacitor. We used that
to estimate the decay time constant and found that to be
\(\tau = 66.57 \pm 0.34\mathrm{~ms}\), very close to the expected value
of \(\tau =66.67\mathrm{~ms}\). (Please replace these numbers with
numbers from your actual experiment and analysis. Then delete this
parenthetical comment.)

\subsection{Description}\label{description}

The intention of this project is to be simple enough that the electrical
circuits are not too confusing or difficult, but also authentic in the
sense that one is collecting real data and doing legitimate analysis of
that data to produce results.

The project invoves two experiments.

\begin{enumerate}
\def\labelenumi{\arabic{enumi}.}
\tightlist
\item
  To drive a resistor network and measure voltages in that network to
  make inferences.
\item
  To monitor the voltage across a discharging capacitor over time to
  infer the value of the time constant.
\end{enumerate}

\section{2. Schematic and Background}\label{schematic-and-background}

Both measurements for this week use the same schemtic:

\begin{figure}[H]

{\centering \pandocbounded{\includegraphics[keepaspectratio]{images/divider.png}}

}

\caption{Schematic Diagram for resistor/capacitor network}

\end{figure}%

In most projects you'll have some design decisions to make to choose
devices and component values to produce a working circuit. In those
cases this is where you'd describe those decisions and explain how to
made them. This week the circuit is provided ``as-is'', so there are no
design decisions to make, and no explanation is needed.

For experiment 1 we'll use the DAC to drive pin 17 with ever increasing
voltages. At the same time we'll be measuring the voltages at pins 15
and 10. So long as the voltages change slowly enough that the capacitor
can fully charge between measurements, the voltage at pin 10 (adc2)
should be proportional to the voltage at pin 15 (adc1) according to the
standard voltage divider relationship:

\[ v_2 = \frac{R_2}{R_1 + R_2} v_1 \]

By driving and measuring \(v_1\) and also measuring \(v_2\), we can
check that the measured ratio is close to the expected value.

We can simulation this experiment using a DC sweep simulation.

\begin{figure}[H]

{\centering \pandocbounded{\includegraphics[keepaspectratio]{images/sim_divider_config.png}}

}

\caption{DC Sweep}

\end{figure}%

and graph the output:

\begin{figure}[H]

{\centering \pandocbounded{\includegraphics[keepaspectratio]{images/sim_divider.png}}

}

\caption{DC Sweep Graph}

\end{figure}%

For experiment 2 we'll charge the capacitor by driving pin-17 high as a
digital output for a long time (much longer than the expected time
constant of the circuit.) Then we'll bring pin-17 to ground and very
quickly measure the voltage across the capacitor (at pin-10) as
frequently as possible capture the decay curve over time. We know the
theoretical decay curve should be exponential:

\[ v_2(t) = v_0 e^{-t/\tau} \]

We can use this model, and the data we collect, to infer the time
constant.

We can configure a transient simulation to check this:

\begin{figure}[H]

{\centering \pandocbounded{\includegraphics[keepaspectratio]{images/sim_decay_config.png}}

}

\caption{decay transient configuration}

\end{figure}%

and graph the output

\begin{figure}[H]

{\centering \pandocbounded{\includegraphics[keepaspectratio]{images/sim_decay.png}}

}

\caption{graph of decay response}

\end{figure}%

\section{3. Function and Data
Collection}\label{function-and-data-collection}

To measure the voltages \(v_1\) and \(v_2\) in the circuit we first had
to discharge the capacitor, by pulling pin-17 low for a long time (3
sec). Then we had to go slowly, raising the voltage on pin-17 and then
waiting to give the capactor time to charge after each voltage change.
Since we expect the time constant of the capactor to be around
\(67\mathrm{~ms}\) we waited \(400\mathrm{~ms}\) between measurements,
just to be sure.

\begin{Shaded}
\begin{Highlighting}[]
\ImportTok{import}\NormalTok{ board }\CommentTok{\# information about the physical microcontroller}
\ImportTok{import}\NormalTok{ analogio }\ImportTok{as}\NormalTok{ aio }\CommentTok{\# analog input and output library}
\ImportTok{import}\NormalTok{ time }\CommentTok{\# standard python time library}

\NormalTok{ADCMAX }\OperatorTok{=} \DecValTok{2}\OperatorTok{**}\DecValTok{16}\OperatorTok{{-}}\DecValTok{1} \CommentTok{\# maximum input/out from ADC or to DAC}

\NormalTok{adc1 }\OperatorTok{=}\NormalTok{ aio.AnalogIn(board.IO15)  }\CommentTok{\# to read voltage from pin 15}
\NormalTok{adc2 }\OperatorTok{=}\NormalTok{ aio.AnalogIn(board.IO10)  }\CommentTok{\# to read voltage from pin 10}
\NormalTok{vfactor }\OperatorTok{=}\NormalTok{ adc1.reference\_voltage}\OperatorTok{/}\NormalTok{ADCMAX }\CommentTok{\# get conversion factor voltage to pin}

\NormalTok{vdac }\OperatorTok{=}\NormalTok{ aio.AnalogOut(board.IO17) }\CommentTok{\# to control voltage on pin 17}

\NormalTok{f }\OperatorTok{=} \BuiltInTok{open}\NormalTok{(}\StringTok{\textquotesingle{}starter.csv\textquotesingle{}}\NormalTok{,}\StringTok{\textquotesingle{}w\textquotesingle{}}\NormalTok{)}
\NormalTok{header }\OperatorTok{=} \StringTok{"j,vdac,v1,v2,time"}


\NormalTok{vdac.value }\OperatorTok{=} \DecValTok{0} \CommentTok{\# set output voltage to zero}
\BuiltInTok{print}\NormalTok{(}\StringTok{"sleeping.... let C fully discharge"}\NormalTok{)}
\BuiltInTok{print}\NormalTok{(}\StringTok{"3"}\NormalTok{)}
\NormalTok{time.sleep(}\DecValTok{1}\NormalTok{)}
\BuiltInTok{print}\NormalTok{(}\StringTok{"2"}\NormalTok{)}
\NormalTok{time.sleep(}\DecValTok{1}\NormalTok{)}
\BuiltInTok{print}\NormalTok{(}\StringTok{"1"}\NormalTok{)}
\NormalTok{time.sleep(}\DecValTok{1}\NormalTok{)}
\BuiltInTok{print}\NormalTok{(}\StringTok{"0 {-}{-}{-} go!"}\NormalTok{)}

\NormalTok{f.write(header }\OperatorTok{+} \StringTok{"}\CharTok{\textbackslash{}n}\StringTok{"}\NormalTok{)}
\BuiltInTok{print}\NormalTok{(header)}

\NormalTok{t0 }\OperatorTok{=}\NormalTok{ time.monotonic\_ns()}

\ControlFlowTok{for}\NormalTok{ i }\KeywordTok{in} \BuiltInTok{range}\NormalTok{(}\DecValTok{0}\NormalTok{,}\DecValTok{255}\NormalTok{,}\DecValTok{5}\NormalTok{): }\CommentTok{\# steps of 5 to save time}
\NormalTok{    time.sleep(}\FloatTok{0.4}\NormalTok{) }\CommentTok{\# let the cap charge....}
\NormalTok{    t }\OperatorTok{=}\NormalTok{ time.monotonic\_ns()}\OperatorTok{{-}}\NormalTok{t0}
\NormalTok{    vout }\OperatorTok{=} \BuiltInTok{int}\NormalTok{((i}\OperatorTok{/}\DecValTok{255}\NormalTok{)}\OperatorTok{*}\NormalTok{ADCMAX)}
\NormalTok{    vdac.value }\OperatorTok{=}\NormalTok{ vout}
\NormalTok{    sval }\OperatorTok{=} \SpecialStringTok{f"}\SpecialCharTok{\{}\NormalTok{i}\SpecialCharTok{\}}\SpecialStringTok{,}\SpecialCharTok{\{}\NormalTok{vout}\OperatorTok{*}\NormalTok{vfactor}\SpecialCharTok{\}}\SpecialStringTok{,}\SpecialCharTok{\{}\NormalTok{adc1}\SpecialCharTok{.}\NormalTok{value}\OperatorTok{*}\NormalTok{vfactor}\SpecialCharTok{\}}\SpecialStringTok{,}\SpecialCharTok{\{}\NormalTok{adc2}\SpecialCharTok{.}\NormalTok{value}\OperatorTok{*}\NormalTok{vfactor}\SpecialCharTok{\}}\SpecialStringTok{,}\SpecialCharTok{\{}\NormalTok{t}\OperatorTok{/}\FloatTok{1e9}\SpecialCharTok{\}}\SpecialStringTok{"}
\NormalTok{    f.write(}\SpecialStringTok{f"}\SpecialCharTok{\{}\NormalTok{sval}\SpecialCharTok{\}}\CharTok{\textbackslash{}n}\SpecialStringTok{"}\NormalTok{)}
    \BuiltInTok{print}\NormalTok{(sval)}

\NormalTok{vdac.value }\OperatorTok{=} \DecValTok{0} \CommentTok{\# let the LED rest...}
\NormalTok{f.close()}
\end{Highlighting}
\end{Shaded}

An example of the data we collected can be seen here:

\begin{longtable}[]{@{}llllll@{}}
\toprule\noalign{}
& j & vdac & v1 & v2 & time \\
\midrule\noalign{}
\endhead
\bottomrule\noalign{}
\endlastfoot
0 & 0 & 0.000000 & 0.079963 & 0.051966 & 0.39978 \\
1 & 5 & 0.064656 & 0.117981 & 0.054987 & 0.80191 \\
2 & 10 & 0.129361 & 0.173976 & 0.078956 & 1.20416 \\
3 & 15 & 0.194067 & 0.233998 & 0.116974 & 1.60587 \\
4 & 20 & 0.258773 & 0.291957 & 0.156956 & 2.00745 \\
\end{longtable}

The raw data looks reasonable:

\pandocbounded{\includegraphics[keepaspectratio]{index_files/figure-latex/ControlAndMeasure-cell-3-output-1.png}}

For the second experiment we used a different program. This time we
charged the capacitor fully, using pin-17 in digital output mode, and
the dropped it suddenly. Then we wrote a tight loop, just measuring
voltages and saving results without printing anything. After the data
was captured, we reformatted and printed the data we had collected. The
code to do this is below.

\begin{Shaded}
\begin{Highlighting}[]
\ImportTok{import}\NormalTok{ analogio }\ImportTok{as}\NormalTok{ aio}
\ImportTok{import}\NormalTok{ digitalio }\ImportTok{as}\NormalTok{ dio}
\ImportTok{import}\NormalTok{ board}
\ImportTok{import}\NormalTok{ time}

\NormalTok{ADCMAX }\OperatorTok{=} \DecValTok{2}\OperatorTok{**}\DecValTok{16}\OperatorTok{{-}}\DecValTok{1}

\NormalTok{adc1 }\OperatorTok{=}\NormalTok{ aio.AnalogIn(board.IO13) }\CommentTok{\# analog ports}
\NormalTok{adc2 }\OperatorTok{=}\NormalTok{ aio.AnalogIn(board.IO15)}
\NormalTok{adc3 }\OperatorTok{=}\NormalTok{ aio.AnalogIn(board.IO10)}
\NormalTok{vfactor }\OperatorTok{=}\NormalTok{ adc1.reference\_voltage}\OperatorTok{/}\NormalTok{ADCMAX}

\NormalTok{dout }\OperatorTok{=}\NormalTok{ dio.DigitalInOut(board.IO17) }\CommentTok{\# digital output}
\NormalTok{dout.direction }\OperatorTok{=}\NormalTok{ dio.Direction.OUTPUT}

\NormalTok{f }\OperatorTok{=} \BuiltInTok{open}\NormalTok{(}\StringTok{\textquotesingle{}decay.csv\textquotesingle{}}\NormalTok{,}\StringTok{\textquotesingle{}w\textquotesingle{}}\NormalTok{)}
\NormalTok{header }\OperatorTok{=} \StringTok{"j,v1,v2,v3,time"}

\NormalTok{data }\OperatorTok{=}\NormalTok{ []}

\NormalTok{dout.value }\OperatorTok{=} \DecValTok{1}
\BuiltInTok{print}\NormalTok{(}\StringTok{"sleeping.... let C fully charge"}\NormalTok{)}
\BuiltInTok{print}\NormalTok{(}\StringTok{"3"}\NormalTok{)}
\NormalTok{time.sleep(}\DecValTok{1}\NormalTok{)}
\BuiltInTok{print}\NormalTok{(}\StringTok{"2"}\NormalTok{)}
\NormalTok{time.sleep(}\DecValTok{1}\NormalTok{)}
\BuiltInTok{print}\NormalTok{(}\StringTok{"1"}\NormalTok{)}
\NormalTok{time.sleep(}\DecValTok{1}\NormalTok{)}
\BuiltInTok{print}\NormalTok{(}\StringTok{"0 {-}{-}{-} go!"}\NormalTok{)}

\NormalTok{t0 }\OperatorTok{=}\NormalTok{ time.monotonic\_ns()}
\NormalTok{dout.value }\OperatorTok{=} \DecValTok{0}

\ControlFlowTok{for}\NormalTok{ i }\KeywordTok{in} \BuiltInTok{range}\NormalTok{(}\DecValTok{0}\NormalTok{,}\BaseNTok{0xff}\NormalTok{,}\BaseNTok{0x2}\NormalTok{): }\CommentTok{\# measure quickly!}
\NormalTok{    time.sleep(}\FloatTok{0.001}\NormalTok{)}
\NormalTok{    data.append((i,adc1.value, adc2.value, adc3.value, time.monotonic\_ns()}\OperatorTok{{-}}\NormalTok{t0))}

\NormalTok{f.write(header }\OperatorTok{+} \StringTok{"}\CharTok{\textbackslash{}n}\StringTok{"}\NormalTok{)}
\BuiltInTok{print}\NormalTok{(header)}

\ControlFlowTok{for}\NormalTok{ i }\KeywordTok{in} \BuiltInTok{range}\NormalTok{(}\BuiltInTok{len}\NormalTok{(data)):}
\NormalTok{    j,v1,v2,v3,t }\OperatorTok{=}\NormalTok{ data[i]}
\NormalTok{    sval }\OperatorTok{=} \SpecialStringTok{f"}\SpecialCharTok{\{}\NormalTok{j}\SpecialCharTok{\}}\SpecialStringTok{,}\SpecialCharTok{\{}\NormalTok{v1}\OperatorTok{*}\NormalTok{vfactor}\SpecialCharTok{\}}\SpecialStringTok{,}\SpecialCharTok{\{}\NormalTok{v2}\OperatorTok{*}\NormalTok{vfactor}\SpecialCharTok{\}}\SpecialStringTok{,}\SpecialCharTok{\{}\NormalTok{v3}\OperatorTok{*}\NormalTok{vfactor}\SpecialCharTok{\}}\SpecialStringTok{,}\SpecialCharTok{\{}\NormalTok{t}\OperatorTok{/}\FloatTok{1e9}\SpecialCharTok{\}}\SpecialStringTok{"}
\NormalTok{    f.write(}\SpecialStringTok{f"}\SpecialCharTok{\{}\NormalTok{sval}\SpecialCharTok{\}}\CharTok{\textbackslash{}n}\SpecialStringTok{"}\NormalTok{)}
    \BuiltInTok{print}\NormalTok{(sval)}
    
\NormalTok{f.close()}
\end{Highlighting}
\end{Shaded}

A sample of the collected data is presented here:

\begin{longtable}[]{@{}llllll@{}}
\toprule\noalign{}
& j & v1 & v2 & v3 & time \\
\midrule\noalign{}
\endhead
\bottomrule\noalign{}
\endlastfoot
0 & 0 & 0.059972 & 0.056951 & 1.95099 & 0.002045 \\
1 & 2 & 0.054987 & 0.055995 & 1.88896 & 0.003906 \\
2 & 4 & 0.053980 & 0.053980 & 1.83095 & 0.005768 \\
3 & 6 & 0.051966 & 0.051966 & 1.77400 & 0.007599 \\
4 & 8 & 0.050959 & 0.049952 & 1.71896 & 0.009460 \\
\end{longtable}

A graph of the raw data is here:

\pandocbounded{\includegraphics[keepaspectratio]{index_files/figure-latex/ControlAndMeasure-cell-5-output-1.png}}

\section{4. Results}\label{results}

Once we have the data we can fit the voltage data to a the simple
voltage divider model to infer the ratio:

We can see this in the fit result from the
\href{CurveFitting.ipynb}{\texttt{CurveFitting.ipynb}} notebook:

\pandocbounded{\includegraphics[keepaspectratio]{index_files/figure-latex/CurveFitting-cell-8-output-1.png}}

Note that these are 5\% resistors, we we'd be unsuprised to see a result
witha variation of around 5\%, or maybe a little more. However the
actual result is within 3\%, so it's pretty good.

We linearized the exponential model by taking the log of both sides,
like so:

\[ \log(v_1(t)) = \log(v_0) - t/\tau \]

which looks a bit like:

\[ y = m x + b \]

but \(y=\log(v_1(t))\), \(x=t\), \(b=\log(v_0)\), and \(m=-1/\tau\).

So we fit the data to a straight line model, similar to the voltage
divider, and we got:

\pandocbounded{\includegraphics[keepaspectratio]{index_files/figure-latex/CurveFitting-cell-13-output-1.png}}

\subsection{Statistics Exercises}\label{statistics-exercises}

The statistics exercises for this project were to perform the curve
fitting for the two experiments. Please see above. 😊

\section{5. Conclusion}\label{conclusion}

We performed two experiments 1. measurements on a voltage divider to
validate the value of the ratio of the bottom resistor to the whole
network and 2. to measure the time constant of an RC network. We found
the ratio of voltages for the voltage divider was: \(0.665\pm 0.001\)
close to the expected value of \(0.667\). We found the time constant of
the RC circuit to be \(tau = 66.57 \pm 0.34\mathrm{~ms}\), very close to
the expected value of \(66.67\mathrm{~ms}\).

The goal of this project was to become famiar with all the tools we'll
be using this semester in Laboratory Instrumentation and to make some
real measurements, and analyze some real data. (Add anything else you
like here, what you learned, suggested improvements, etc. Be sure to
delete this parenthetical remark in your final report.)




\end{document}
